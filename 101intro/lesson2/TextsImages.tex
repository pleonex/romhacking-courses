\begin{frame}{Contenido}
    \begin{columns}
    \begin{column}{0.5\textwidth}
        \uncover<2->{\large Textos}
        \begin{itemize}
            \item<2-> Codificaciones
            \item<2-> Formatos
        \end{itemize}
        \uncover<2->{Tipografías}
        \visible<2->{\includegraphics[width=0.8\textwidth]{imgs/example_text.png}}
    \end{column}
    \hfill
    \begin{column}{0.5\textwidth}
        \uncover<3->{\large Imágenes}
        \begin{itemize}
            \item<3-> Fondos
            \item<3-> Sprites
            \item<3-> Texturas
        \end{itemize}
        \vfill{}
        \visible<3->{\includegraphics[width=0.8\textwidth]{imgs/example_img.png}}
    \end{column}
    \end{columns}
\end{frame}

\section{Textos}
\begin{frame}{Naturaleza del texto}
    \centering\Large ¿Cómo guardamos texto de forma digital?
    \includegraphics[width=0.75\textwidth]{imgs/example_text.png}
\end{frame}

\subsection{Codificaciones}
\begin{frame}{Codificación de caracteres}
    \centering
    \only<1>{
        \begin{block}{}
            Es el método que permite convertir un carácter de un lenguaje natural en un símbolo de otro sistema de representación aplicando reglas de codificación. [Wikipedia]
        \end{block}
    }
    \only<2>{\includegraphics[width=\textwidth]{imgs/ascii-table.png}}
\end{frame}

\begin{frame}{ASCII}
    \begin{block}{}
        \centering Codifica caracteres del alfabeto latino en 7 bits.
    \end{block}
    \begin{columns}
    \begin{column}{0.40\textwidth}
        \includegraphics[trim={1616px 0 0 0},clip,width=\textwidth]{imgs/ascii-table.png}
    \end{column}
    \begin{column}{0.55\textwidth}
        \centering
        \includegraphics[width=\textwidth]{imgs/text_hex.png} \\
        \Huge \textdownarrow \\
        \includegraphics[width=0.7\textwidth]{imgs/text_hex1.png}
    \end{column}
    \end{columns}
\end{frame}

\begin{frame}{Latin-1 (ISO 8859-1)}
    \begin{block}{}
        Codificación extendida de ASCII. Utiliza 8 bits añadiendo 128 caracteres necesarios para las lenguas europeas.
    \end{block}
    \centering\includegraphics[width=0.75\textwidth]{imgs/latin1.png}
\end{frame}

\begin{frame}{Unicode}
    \begin{block}{}
        Estándar universal de codificación de caracteres para la mayoría de lenguas (incluidas las muertas). La última version 6.0 incluye 109.449 caracteres.
    \end{block}

    \begin{itemize}
        \item<2-> Unicode es solo una tabla, no especifica la codificación.
        \item<3-> Codificaciones para unicode:
        \begin{itemize}
            \item<4-> UTF-8: codificación de 8 bits de longitud variable.\\ \texttt{'A' = 41h, '}\visible<4->{\includegraphics[height=7px]{imgs/rare_char.png}}\texttt{' = F0 A0 9C 8E}
            \item<5-> UTF-16: codificación de 16 bits de longitud variable. \\ \texttt{'A' = 0041, '}\visible<5->{\includegraphics[height=7px]{imgs/rare_char.png}}\texttt{' = D841 DF0E}
            \item<6-> UTF-32: codificación de 8 bits de longitud fija. \\ \texttt{'A' = 00000041, '}\visible<6->{\includegraphics[height=7px]{imgs/rare_char.png}}\texttt{' = 0002070E}
        \end{itemize}
    \end{itemize}
\end{frame}

\begin{frame}{Shift Jis (CP 932)}
    \begin{block}{}
        Codificación de longitud variable (1 o 2 bytes) para caracteres japoneses. Incluye ASCII.
    \end{block}
    \includefigure{Tabla para caracteres con primer byte \texttt{0x82}}{imgs/shiftjis_table82.png}{0.55}
\end{frame}

\begin{frame}{Ejemplos en archivos}
    \centering\Large
    \only<1>{¿? \\}
    \only<2->{UTF-16 \\}
    \includegraphics[width=0.75\textwidth]{imgs/utf16.png} \\
    \only<3>{¿? \\}
    \only<4->{Shift Jis \\}
    \visible<3->{\includegraphics[width=0.75\textwidth]{imgs/shiftjis.png}}
\end{frame}

\begin{frame}{Caracteres no imprimibles}
    % \n, \r, \t and \0
    % Also in scripts to do things
\end{frame}

\begin{frame}{Tablas}
\end{frame}

\begin{frame}{Tablas}
    % TODO: Get decoded Pokemon text file and figure out table.
\end{frame}

\subsection{Formatos}
\begin{frame}{Punteros (\textit{offsets})}
    \begin{block}{}
        \centering{}Número enteros que indican la posición del texto.
    \end{block}
    \vfill{}
    \uncover<2->{\large Tipos:}
    \begin{itemize}
        \item<3-> \textit{Absoluto}: desde el comienzo del archivo.
        \item<4-> Relativo: desde otra posición
        \begin{itemize}
            \item Relativo a una sección del fichero.
            \item Relativo a la posición actual.
        \end{itemize}
    \end{itemize}
\end{frame}

\begin{frame}{Punteros (\textit{offsets})}
    \begin{tikzpicture}
        \node[anchor=south west,inner sep=0] (image) at (0,0)
        {\includegraphics[width=\textwidth]{imgs/pointers.png}};
            \begin{scope}[x={(image.south east)},y={(image.north west)}]
                \draw[black,thick] (0.775,0) -- (0.775,1);
                \draw<2-4>[red,semithick] (0.13,0.9) rectangle (1,0.57);
                \draw<2-4>[blue,semithick] (0.13,0.565) rectangle (1,0.0);
                \draw<3,4>[blue,semithick] (0.13,0.565) rectangle (0.446,0.495);
                \draw<4->[blue,line width=2pt,->] (0.446,0.65) -- (0.446,0.55);
                \draw<5,6>[red,semithick] (0.13,0.81) rectangle (0.29,0.745);
                \draw<6>[blue,semithick] (0.765,0.565) -- (0.53,0.565) -- (0.53,0.495) -- (0.765,0.495);
                \draw<6>[blue,semithick] (1,0.565) -- (0.913,0.565) -- (0.913,0.495) -- (1,0.495);
                \draw<6>[blue,semithick] (0.78,0.49) -- (0.835,0.49) -- (0.835,0.41) -- (0.78,0.41);
                \draw<6>[blue,semithick] (0.13,0.49) -- (0.29,0.49) -- (0.29,0.41) -- (0.13,0.41);
                \draw<7,8>[red,semithick] (0.45,0.81) rectangle (0.60,0.745);
                \draw<8,9>[blue,semithick] (0.13, 0) -- (0.13, 0.4) -- (0.29, 0.4) -- (0.29,0.49) -- (0.765, 0.49) -- (0.765, 0);
                \draw<8,9>[blue,semithick] (0.78, 0) -- (0.78, 0.4) -- (0.835, 0.4) -- (0.835, 0.49) -- (1, 0.49) -- (1, 0);
                \draw<9>[green,semithick] (0.21, 0.49) rectangle (0.29, 0.4);
                \draw<9>[green,semithick] (0.37, 0.4) rectangle (0.445, 0.32);
                \draw<9>[green,semithick] (0.37, 0.24) rectangle (0.445, 0.16);
            \end{scope}
    \end{tikzpicture}
\end{frame}

\section{Imágenes}
\subsection{Imágenes de fondo}
\begin{frame}{Colores}
\end{frame}

\begin{frame}{Paletas}
\end{frame}

\begin{frame}{Píxeles}
\end{frame}

\begin{frame}{Codificaciones}
\end{frame}

\begin{frame}{Agrupación de píxeles}
\end{frame}

\begin{frame}{Compresión}
\end{frame}

\subsection{Sprites}
\begin{frame}{Formatos base}
\end{frame}

\begin{frame}{Celdas}
\end{frame}

\begin{frame}{Reordenando el puzzle}
\end{frame}

\begin{frame}{Bancos}
\end{frame}

\begin{frame}{Animaciones}
\end{frame}

\subsection{3D}
\begin{frame}{Texturas}
\end{frame}

\begin{frame}{Modelos 3D}
\end{frame}

\section{Media}
\subsection{Otros formatos}
\begin{frame}{Tipografías}
\end{frame}

\begin{frame}{Formatos de audio}
\end{frame}

\begin{frame}{Audios de onda}
\end{frame}

\begin{frame}{Audios MIDI}
\end{frame}

\section{Programas de edición}
\subsection{Pokémon}
\begin{frame}{PokeText}
\end{frame}

\subsection{New Super Mario Bros DS}
\begin{frame}{NSMB}
\end{frame}

\subsection{Super Mario 64 DS}
\begin{frame}{Super Mario 64 Editor}
\end{frame}

\subsection{Ni no kuni}
\begin{frame}{NinoCompiler}
\end{frame}

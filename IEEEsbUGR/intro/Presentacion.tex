% Copyright (c) 2017 Benito Palacios Sánchez - All Rights Reserved.
% Esta obra está licenciada bajo la Licencia Creative Commons Atribución 4.0
% Internacional. Para ver una copia de esta licencia, visita
% http://creativecommons.org/licenses/by/4.0/.

% Template
\documentclass[usenames,dvipsnames]{beamer}
\pdfoutput=1

% Notes. Uncomment to view the notes.
%\setbeameroption{show notes}
%\setbeamertemplate{note page}[plain]    % Remove the note page style

% Load this package to allow load many big packages.
\usepackage{etex}

% Font
\usepackage[T1]{fontenc}        % Output font
\usepackage[utf8]{inputenc}     % Input encoding
\usepackage[spanish]{babel}     % For Spanish texts
\usepackage{FiraSans}           % For FiraSans beauty fonts

% Theme
\usetheme{Darmstadt}
\usecolortheme{whale}

% Other packages
\usepackage{xcolor}             % For color in text
\usepackage{url}                % For links
\usepackage{pifont}             % For tick symbol
\usepackage{graphicx}           % For graphics
\usepackage{epstopdf}           % For EPS graphics in Windows
\usepackage{multimedia}         % For media
\usepackage{verbatim}           % For non-parsed text blocks
\usepackage{tikz}               % To draw over images
\usepackage{ctable}             % For tables
\usepackage{textcomp}           % For text arrows.
\usepackage{listings}           % For blocks of code
\lstset{language=[Sharp]C,basicstyle=\scriptsize\ttfamily, keywordstyle=\scriptsize\color{blue}\ttfamily}

% My package
\usepackage{../Layout}

% Information about author and document
\title{Introducción al ROM Hacking}
\subtitle{Primeros pasos}
\date[Marzo de 2017]{24 de marzo de 2017}
\author{Benito Palacios}
\authortitle{}
\authoremail{benito.palaciossanchez.es@ieee.org}
\institute[IEEE SB UGR]{Rama estudiantil de IEEE en la UGR}
\titlelogo{../logo.png}

% Add a little logo in the corner of the slides
\pgfdeclareimage[height=0.5cm]{logo-mini}{../logo_mini.png}
\logo{\pgfuseimage{logo-mini}}

\begin{document}

    % Title page
    {
    \usebackgroundtemplate{
        \includegraphics[width=\paperwidth]{../background.jpg}}
    \begin{frame}[plain]
        \titlepage{}
    \end{frame}
    }

    % Content
    \include{Introduccion}
    \section{Hello World ROM Hacking!}
\subsection{Conceptos}
\begin{frame}[fragile]{Números hexadecimales}
    \begin{uncoverenv}<2->Decimal: 0 1 2 3 4 5 6 7 8 9
    \begin{lstlisting}
    0 1 2 3 4 5 6 7 8 9 10 11 12 13 ...\end{lstlisting}\end{uncoverenv}

    \begin{uncoverenv}<3->Binario: 0 1
    \begin{lstlisting}
    0 1 10 11 100 101 110 111 1000 ...
    0 1  2  3   4   5   6   7    8\end{lstlisting}\end{uncoverenv}

    \begin{uncoverenv}<4->Octal: 0 1 2 3 4 5 6 7
    \begin{lstlisting}
    0 1 2 3 4 5 6 7 10 11 12 13 14 ...
    0 1 2 3 4 5 6 7  8  9 10 11 12\end{lstlisting}\end{uncoverenv}

    \begin{uncoverenv}<5->Hexadecimal: 0 1 2 3 4 5 6 7 8 9 A B C D E F
    \begin{lstlisting}
    0 1 2 3 4 5 6 7 8 9  A  B  C  D  E  F 10 11 12 ...
    0 1 2 3 4 5 6 7 8 9 10 11 12 13 14 15 16 17 18\end{lstlisting}\end{uncoverenv}
\end{frame}

\begin{frame}{Números hexadecimales}
    \begin{itemize}
        \item<1-> Prefijo: \texttt{0x} -> \texttt{0xA, 0xFB, 0xCA, 0xFE}

        \item<2-> Representación de tipos:
        \begin{itemize}
            \footnotesize
            \item<3-> 1: \texttt{[0, 15] 0xC (12)} -> \textonehalf~byte, 4 bits
            \item<4-> 2: \texttt{[0, 255] 0xC0 (192)} -> 1 byte, 8 bits
            \item<5-> 4: \texttt{[0, 65,535] 0x0200 (512)} -> 2 bytes, 16 bits, ushort, WORD
            \item<6-> 8: \texttt{[0, 4,294,967,295] 0xB7000000 (3,070,230,528)} -> \\ 4 bytes, 32 bits, uint, DWORD
        \end{itemize}
        \normalsize

        \item<7-> Operaciones a nivel de bits:
        \begin{itemize}
            \footnotesize
            \item<9-> Filtrar/Máscaras: \texttt{0xAFC2 AND 0xF800 = 0xA800}
            \item<10-> Formar valores: \texttt{0xB7000000 OR 0xCADB00 = 0xB70ADB00}
            \item<11-> Cifrar: \texttt{0xCAFE XOR 0xAAAA = 0x6054, NOT 0xBEBE = 0x3501}
            \item<12-> Desplazamientos: \texttt{0xC2 << 4 = 0xC20, 0xA800 >> 8 = 0xA8}
        \end{itemize}
    \end{itemize}
    \visible<8->{\includegraphics[width=\textwidth]{bitop.png}}
\end{frame}


\begin{frame}{Endianness}
    \begin{block}{}
        Orden en el que se guardan los bytes que forman valores mayores a 8 bits (ushort, uint, ulong, \ldots). MSB \textrightarrow LSB.
    \end{block}
    \centering{}Big Endian:
    \includegraphics[width=0.9\textwidth]{big_endian.png}
    \vfill
    Little Endian (más común):
    \includegraphics[width=0.9\textwidth]{little_endian.png}
\end{frame}

\subsection{Investigando un juego}
\begin{frame}{Especificación de juegos de NDS}
\end{frame}

\begin{frame}{Cabecera de los juegos}
\end{frame}

\begin{frame}{Buscar los textos en hexadecimal}
\end{frame}

\begin{frame}{Tinke}
    % Abrir tinke, acciones típicas y significado de iconos
\end{frame}

\begin{frame}{Tinke}
    % Buscar texto desde Tinke
\end{frame}

\subsection{Editar juegos}
\begin{frame}{Modificando texto}
    % Juego: Layton
    % Extraer archivo
    % Editarlo con Notepad++ o Atom
    % Importar archivo
    % Generar ROM
    % Probar en DeSmuME
\end{frame}

\subsection{Distribuyamos los cambios}
\begin{frame}{Legalidad}
\end{frame}

\begin{frame}{Parches}
\end{frame}

\begin{frame}{xDelta}
\end{frame}

    \section{Reto 1}
\begin{frame}{Reto 1}
    \Huge\centering\textbf{CHALLENGE TIME!}
\end{frame}

\subsection{Puzles en Profesor Layton}
\begin{frame}{Objetivo}

    {\Large\bfseries Modifica un puzle}
    \begin{columns}
    \begin{column}{0.5\textwidth}
        \begin{itemize}
            \item<1-> Descripción (\textit{qtext})
            \item<2-> Pistas (\textit{qtext})
            \item<3-> Solución (\textit{script/qscript})
            \item<4-> Confirmación (\textit{qtext})
        \end{itemize}
    \end{column}
    \begin{column}{0.5\textwidth}
        \only<1>{\includegraphics[width=0.7\textwidth]{imgs/reto1_1.png}}
        \only<2>{\includegraphics[width=0.7\textwidth]{imgs/reto1_2.png}}
        \only<3>{\includegraphics[width=0.7\textwidth]{imgs/reto1_3.png}}
        \only<4>{\includegraphics[width=0.7\textwidth]{imgs/reto1_4.png}}
    \end{column}
    \end{columns}

    \note[itemize]{
        \item Primeros 4 bytes es el tamaño del archivo
        \item Se compone de múltiples comandos + argumentos
        \item Después de \texttt{0x0000} va el ID del comando
        \item El formato de los argumentos es tipo + valor
        \item El tipo \texttt{0x0001} es un entero de 32 bits
        \item El tipo \texttt{0x0003} es una cadena de caracteres de longitud variable
        \item Los scripts terminan con \texttt{0x000C}
    }
\end{frame}

    \begin{frame}{Contenido}
    \begin{columns}
    \begin{column}{0.5\textwidth}
        \uncover<2->{\large Textos}
        \begin{itemize}
            \item<2-> Codificaciones
            \item<2-> Formatos
        \end{itemize}
        \uncover<2->{Tipografías}
        \visible<2->{\includegraphics[width=0.8\textwidth]{imgs/example_text.png}}
    \end{column}
    \hfill
    \begin{column}{0.5\textwidth}
        \uncover<3->{\large Imágenes}
        \begin{itemize}
            \item<3-> Fondos
            \item<3-> Sprites
            \item<3-> Texturas
        \end{itemize}
        \vfill{}
        \visible<3->{\includegraphics[width=0.8\textwidth]{imgs/example_img.png}}
    \end{column}
    \end{columns}
\end{frame}

\section{Textos}
\begin{frame}{Naturaleza del texto}
    \centering\Large ¿Cómo guardamos texto de forma digital?
    \includegraphics[width=0.75\textwidth]{imgs/example_text.png}
\end{frame}

\subsection{Codificaciones}
\begin{frame}{Codificación de caracteres}
    \centering
    \only<1>{
        \begin{block}{}
            Es el método que permite convertir un carácter de un lenguaje natural en un símbolo de otro sistema de representación aplicando reglas de codificación. [Wikipedia]
        \end{block}
    }
    \only<2>{\includegraphics[width=\textwidth]{imgs/ascii-table.png}}
\end{frame}

\begin{frame}{ASCII}
    \begin{block}{}
        \centering Codifica caracteres del alfabeto latino en 7 bits.
    \end{block}
    \begin{columns}
    \begin{column}{0.40\textwidth}
        \includegraphics[trim={1616px 0 0 0},clip,width=\textwidth]{imgs/ascii-table.png}
    \end{column}
    \begin{column}{0.55\textwidth}
        \centering
        \includegraphics[width=\textwidth]{imgs/text_hex.png} \\
        \Huge \textdownarrow \\
        \includegraphics[width=0.7\textwidth]{imgs/text_hex1.png}
    \end{column}
    \end{columns}
\end{frame}

\begin{frame}{Latin-1 (ISO 8859-1)}
    \begin{block}{}
        Codificación extendida de ASCII. Utiliza 8 bits añadiendo 128 caracteres necesarios para las lenguas europeas.
    \end{block}
    \centering\includegraphics[width=0.75\textwidth]{imgs/latin1.png}
\end{frame}

\begin{frame}{Unicode}
    \begin{block}{}
        Estándar universal de codificación de caracteres para la mayoría de lenguas (incluidas las muertas). La última version 6.0 incluye 109.449 caracteres.
    \end{block}

    \begin{itemize}
        \item<2-> Unicode es solo una tabla, no especifica la codificación.
        \item<3-> Codificaciones para unicode:
        \begin{itemize}
            \item<4-> UTF-8: codificación de 8 bits de longitud variable.\\ \texttt{'A' = 41h, '}\visible<4->{\includegraphics[height=7px]{imgs/rare_char.png}}\texttt{' = F0 A0 9C 8E}
            \item<5-> UTF-16: codificación de 16 bits de longitud variable. \\ \texttt{'A' = 0041, '}\visible<5->{\includegraphics[height=7px]{imgs/rare_char.png}}\texttt{' = D841 DF0E}
            \item<6-> UTF-32: codificación de 8 bits de longitud fija. \\ \texttt{'A' = 00000041, '}\visible<6->{\includegraphics[height=7px]{imgs/rare_char.png}}\texttt{' = 0002070E}
        \end{itemize}
    \end{itemize}
\end{frame}

\begin{frame}{Shift Jis (CP 932)}
    \begin{block}{}
        Codificación de longitud variable (1 o 2 bytes) para caracteres japoneses. Incluye ASCII.
    \end{block}
    \includefigure{Tabla para caracteres con primer byte \texttt{0x82}}{imgs/shiftjis_table82.png}{0.55}
\end{frame}

\begin{frame}{Ejemplos en archivos}
    \centering\Large
    \only<1>{¿? \\}
    \only<2->{UTF-16 \\}
    \includegraphics[width=0.75\textwidth]{imgs/utf16.png} \\
    \only<3>{¿? \\}
    \only<4->{Shift Jis \\}
    \visible<3->{\includegraphics[width=0.75\textwidth]{imgs/shiftjis.png}}
\end{frame}

\begin{frame}{Caracteres no imprimibles}
    % \n, \r, \t and \0
    % Also in scripts to do things
\end{frame}

\begin{frame}{Tablas}
\end{frame}

\begin{frame}{Tablas}
    % TODO: Get decoded Pokemon text file and figure out table.
\end{frame}

\subsection{Formatos}
\begin{frame}{Punteros (\textit{offsets})}
    \begin{block}{}
        \centering{}Número enteros que indican la posición del texto.
    \end{block}
    \vfill{}
    \uncover<2->{\large Tipos:}
    \begin{itemize}
        \item<3-> \textit{Absoluto}: desde el comienzo del archivo.
        \item<4-> Relativo: desde otra posición
        \begin{itemize}
            \item Relativo a una sección del fichero.
            \item Relativo a la posición actual.
        \end{itemize}
    \end{itemize}
\end{frame}

\begin{frame}{Punteros (\textit{offsets})}
    \begin{tikzpicture}
        \node[anchor=south west,inner sep=0] (image) at (0,0)
        {\includegraphics[width=\textwidth]{imgs/pointers.png}};
            \begin{scope}[x={(image.south east)},y={(image.north west)}]
                \draw[black,thick] (0.775,0) -- (0.775,1);
                \draw<2-4>[red,semithick] (0.13,0.9) rectangle (1,0.57);
                \draw<2-4>[blue,semithick] (0.13,0.565) rectangle (1,0.0);
                \draw<3,4>[blue,semithick] (0.13,0.565) rectangle (0.446,0.495);
                \draw<4->[blue,line width=2pt,->] (0.446,0.65) -- (0.446,0.55);
                \draw<5,6>[red,semithick] (0.13,0.81) rectangle (0.29,0.745);
                \draw<6>[blue,semithick] (0.765,0.565) -- (0.53,0.565) -- (0.53,0.495) -- (0.765,0.495);
                \draw<6>[blue,semithick] (1,0.565) -- (0.913,0.565) -- (0.913,0.495) -- (1,0.495);
                \draw<6>[blue,semithick] (0.78,0.49) -- (0.835,0.49) -- (0.835,0.41) -- (0.78,0.41);
                \draw<6>[blue,semithick] (0.13,0.49) -- (0.29,0.49) -- (0.29,0.41) -- (0.13,0.41);
                \draw<7,8>[red,semithick] (0.45,0.81) rectangle (0.60,0.745);
                \draw<8,9>[blue,semithick] (0.13, 0) -- (0.13, 0.4) -- (0.29, 0.4) -- (0.29,0.49) -- (0.765, 0.49) -- (0.765, 0);
                \draw<8,9>[blue,semithick] (0.78, 0) -- (0.78, 0.4) -- (0.835, 0.4) -- (0.835, 0.49) -- (1, 0.49) -- (1, 0);
                \draw<9>[green,semithick] (0.21, 0.49) rectangle (0.29, 0.4);
                \draw<9>[green,semithick] (0.37, 0.4) rectangle (0.445, 0.32);
                \draw<9>[green,semithick] (0.37, 0.24) rectangle (0.445, 0.16);
            \end{scope}
    \end{tikzpicture}
\end{frame}

\section{Imágenes}
\subsection{Imágenes de fondo}
\begin{frame}{Colores}
\end{frame}

\begin{frame}{Paletas}
\end{frame}

\begin{frame}{Píxeles}
\end{frame}

\begin{frame}{Codificaciones}
\end{frame}

\begin{frame}{Agrupación de píxeles}
\end{frame}

\begin{frame}{Compresión}
\end{frame}

\subsection{Sprites}
\begin{frame}{Formatos base}
\end{frame}

\begin{frame}{Celdas}
\end{frame}

\begin{frame}{Reordenando el puzzle}
\end{frame}

\begin{frame}{Bancos}
\end{frame}

\begin{frame}{Animaciones}
\end{frame}

\subsection{3D}
\begin{frame}{Texturas}
\end{frame}

\begin{frame}{Modelos 3D}
\end{frame}

\section{Media}
\subsection{Otros formatos}
\begin{frame}{Tipografías}
\end{frame}

\begin{frame}{Formatos de audio}
\end{frame}

\begin{frame}{Audios de onda}
\end{frame}

\begin{frame}{Audios MIDI}
\end{frame}

\section{Programas de edición}
\subsection{Pokémon}
\begin{frame}{PokeText}
\end{frame}

\subsection{New Super Mario Bros DS}
\begin{frame}{NSMB}
\end{frame}

\subsection{Super Mario 64 DS}
\begin{frame}{Super Mario 64 Editor}
\end{frame}

\subsection{Ni no kuni}
\begin{frame}{NinoCompiler}
\end{frame}

    \section{Reto 2}
\begin{frame}{Reto 2}
    \Huge\centering\textbf{CHALLENGE TIME!}
\end{frame}

\subsection{The Legend of Zelda: Phantom Hourglass}
\begin{frame}[t]{Contenido sorpresa}
    \begin{columns}
    \begin{column}[T]{0.5\textwidth}
        \centering
        \visible<2->{¿\textit{/Test/picture.narc}?\\\vspace{5pt}}
        \visible<3->{\includegraphics[width=\textwidth]{imgs/zelda_dog.png}}
    \end{column}
    \hfill
    \begin{column}[T]{0.5\textwidth}
        \centering
        \visible<4->{¿\textit{/Test/BgMap.narc}?\\\vspace{5pt}}
        \visible<5->{\includegraphics[width=0.5\textwidth]{imgs/zelda_batch.png}}
    \end{column}
    \end{columns}
\end{frame}

\begin{frame}{Historia}
    \begin{columns}
    \begin{column}{0.5\textwidth}
        \begin{itemize}
            \item<1-> Modificar cuarto diálogo.
            \item<2-> Modificar imagen.
        \end{itemize}
        \vfill{}
        \hfill{}
        \visible<3->{\includegraphics[width=0.4\textwidth]{imgs/reto2_3.png}}
    \end{column}
    \begin{column}{0.5\textwidth}
        \visible<1->{\includegraphics[width=0.4\textwidth]{imgs/reto2_1.png}}
        \hfill{}
        \visible<2->{\includegraphics[width=0.4\textwidth]{imgs/reto2_2.png}}
        \vfill{}
        \begin{itemize}
            \item<3-> Modificar tipografía.
        \end{itemize}
    \end{column}
    \end{columns}
\end{frame}

\begin{frame}{Historia: Punteros}
    \begin{columns}
    \begin{column}{0.3\textwidth}
        \includegraphics[width=\textwidth]{imgs/reto2_1.png}
    \end{column}
    \begin{column}{0.7\textwidth}
        Modificar cuarto diálogo.
        \footnotesize
        \begin{itemize}
            \item Ruta: \textit{/Spanish/Message/demo.bmg}.
            \item Punteros en la sección \texttt{INF1} (cabecera: 16 B).
            \item Hay \texttt{0x08} bytes por texto.
            \item Primeros \texttt{0x04} bytes es el puntero.
            \item Puntero \textit{i}: $Offset_{INF1} + 16 + i*8=48h$.
            \item Punteros relativos a \texttt{0x0AE8}.
            \item Ignorar 6 bytes después de \texttt{001Ah}.
            \item Añadir pausa al final con: \\
                  \texttt{1A 00 08 01 0E 00 7D 00}.
        \end{itemize}
    \end{column}
    \end{columns}
\end{frame}

\begin{frame}{Historia: Imágenes}
    \begin{columns}
    \begin{column}{0.7\textwidth}
        Modificar imagen
        \footnotesize
        \begin{wideitemize}
            \item Ruta: \textit{/Event/Kamishibai/kami1/kami1-01}.
            \item Seleccionar \textit{Replace palette}.
            \item Importar imagen con mismas dimensiones:
            \begin{itemize}
                \item Ancho: \texttt{256} píxeles.
                \item Alto: \texttt{192} píxeles.
            \end{itemize}
        \end{wideitemize}
    \end{column}
    \begin{column}{0.3\textwidth}
        \includegraphics[width=\textwidth]{imgs/reto2_2.png}
    \end{column}
    \end{columns}
\end{frame}

\begin{frame}{Historia: Tipografía}
    \begin{columns}
    \begin{column}{0.3\textwidth}
        \includegraphics[width=\textwidth]{imgs/reto2_3.png}
    \end{column}
    \begin{column}{0.7\textwidth}
        Modificar tipografía
        \footnotesize
        \begin{wideitemize}
            \item Cambiar primer texto por:\\
                  $\int2x dx=x^2+K$
            \item Tipografía: \textit{/Font/zeldaDS\_15.nftr}.
            \item Reemplazar caracteres japoneses por: $\int$ y $^2$
        \end{wideitemize}
    \end{column}
    \end{columns}
\end{frame}


\end{document}

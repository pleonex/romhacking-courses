\section{Hello World!}
\begin{frame}{Hello World!}
    \Huge\centering\textbf{ROM HACKING TIME!}
\end{frame}

\subsection{Conceptos}
\begin{frame}[fragile]{Números hexadecimales}
    \begin{uncoverenv}<2->Decimal: 0 1 2 3 4 5 6 7 8 9
    \begin{lstlisting}
    0 1 2 3 4 5 6 7 8 9 10 11 12 13 ...\end{lstlisting}\end{uncoverenv}

    \begin{uncoverenv}<3->Binario: 0 1
    \begin{lstlisting}
    0 1 10 11 100 101 110 111 1000 ...
    0 1  2  3   4   5   6   7    8\end{lstlisting}\end{uncoverenv}

    \begin{uncoverenv}<4->Octal: 0 1 2 3 4 5 6 7
    \begin{lstlisting}
    0 1 2 3 4 5 6 7 10 11 12 13 14 ...
    0 1 2 3 4 5 6 7  8  9 10 11 12\end{lstlisting}\end{uncoverenv}

    \begin{uncoverenv}<5->Hexadecimal: 0 1 2 3 4 5 6 7 8 9 A B C D E F
    \begin{lstlisting}
    0 1 2 3 4 5 6 7 8 9  A  B  C  D  E  F 10 11 12 ...
    0 1 2 3 4 5 6 7 8 9 10 11 12 13 14 15 16 17 18\end{lstlisting}\end{uncoverenv}
\end{frame}

\begin{frame}[fragile]{Números hexadecimales}
    \begin{wideitemize}
        \item<1-> Prefijo \texttt{0x} o sufijo \texttt{h} \\
        \texttt{0xA, FBh, 0xCA, FEh}

        \item<2-> Fácil representación de enteros:
    \end{wideitemize}
    \visible<2->{\footnotesize\ctable[]{cccccl}{}{                          \FL
        \# & Rango & Ejemplo & Bytes & Bits & Otros nombres                 \ML
        1 & \texttt{[0, 15]} & \texttt{0xC} & \textonehalf & 4 &            \NN
        2 & \texttt{[0, 255]} & \texttt{0xC0} & 1 & 8 & byte                \NN
        4 & \texttt{[0, 65,535]} & \texttt{0x0200} & 2 & 16 & ushort, WORD  \NN
        8 & \texttt{[0, 4,294,967,295]} & \texttt{0xB7000000} & 4 & 32 & uint, DWORD \LL
    }}
\end{frame}

\subsection{Editar juegos}
\begin{frame}{Modificando textos}
    \begin{columns}
    \begin{column}{0.5\textwidth}
        \begin{enumerate}
            \item<1-> Localizar texto a modificar
            \item<2-> Abrir juego en Tinke
            \item<3-> Localizar textos
            \item<4-> Extraer archivo
            \item<5-> Modificar con editor
            \item<6-> Importar archivo
            \item<7-> Generar ROM
            \item<8-> Probar en DeSmuME
        \end{enumerate}
    \end{column}
    \hfill
    \begin{column}{0.5\textwidth}
        \centering
        \only<1>{\includegraphics[width=0.7\textwidth]{imgs/mod1.png}}
        \only<2>{\includegraphics[width=\textwidth]{imgs/mod2.png}}
        \only<3>{\includegraphics[width=\textwidth]{imgs/mod3.png}}
        \only<4>{\includegraphics[width=\textwidth]{imgs/mod4.png}}
        \only<5>{\includegraphics[width=\textwidth]{imgs/mod5.png}}
        \only<6>{\includegraphics[width=\textwidth]{imgs/mod6.png}}
        \only<7>{\includegraphics[width=\textwidth]{imgs/mod7.png}}
        \only<8>{\includegraphics[width=0.7\textwidth]{imgs/mod8.png}}
    \end{column}
    \end{columns}
\end{frame}

\subsection{Publicar cambios}
\begin{frame}{Parches}
    \begin{columns}
    \begin{column}{0.6\textwidth}
        \small
        \begin{wideitemize}
            \item<1-> Solo contienen las modificaciones
            \item<2-> \textbf{No subir el juego modificado}
            \item<3-> Tamaño pequeño
            \item<4-> Formatos comunes: xDelta
        \end{wideitemize}
    \end{column}
    \begin{column}{0.4\textwidth}
        \includegraphics[width=\textwidth]{imgs/ninopatcher.png}
    \end{column}
    \end{columns}
\end{frame}

\subsection{Programas de edición}
\begin{frame}{Pokémon: Advanced Map}
    \centering
    \includegraphics[width=\textwidth]{imgs/advanced_map.png} \\
    Proyecto: \url{http://ampage.no-ip.info/}
\end{frame}

\begin{frame}{Pokémon: Spiky's DS Map Editor}
    \centering
    \includegraphics[width=0.85\textwidth]{imgs/sdme.png} \\
    Proyecto: \url{https://github.com/MarcRiera/SDSME}
\end{frame}

\begin{frame}{NSMB Editor}
    \centering
    \includegraphics[width=\textwidth]{imgs/nsmbe.png} \\
    Proyecto: \url{https://github.com/Dirbaio/NSMB-Editor}
\end{frame}

\begin{frame}{Super Mario 64 Editor}
    \centering
    \includegraphics[width=0.9\textwidth]{imgs/sm64ds.png} \\
    Descarga: \url{http://www.romhacking.net/utilities/764}
\end{frame}

\begin{frame}{Ninokuni}
    \begin{tikzpicture}
        \node<1-> (img1) at (0,0) {\includegraphics[width=\textwidth]{imgs/ninoscritor.png}};

        \node<2-> (img2) at (-1.2, 0.7) {\includegraphics[width=0.75\textwidth]{imgs/ninoxml.png}};
        \node<3-> (img3) at (2,-1) {\includegraphics[width=0.3\textwidth]{imgs/ninoimg.png}\huge\textrightarrow\includegraphics[width=0.3\textwidth]{imgs/ninoimg_trans.png}};
        \node<4-> (img4) at (0,0) {\includegraphics[width=0.5\textwidth]{imgs/ninocompiler.png}};
    \end{tikzpicture}

    \visible<4->{\centering
    Proyecto: \url{http://gradienwords.com} \\
    GitHub: \url{https://github.com/pleonex/Ninokuni}}
\end{frame}

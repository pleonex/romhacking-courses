\section{Reto 1}
\begin{frame}{Reto 1}
    \Huge\centering\textbf{CHALLENGE TIME!}
\end{frame}

\subsection{Puzles en Profesor Layton}
\begin{frame}{Objetivo}

    {\Large\bfseries Modifica un puzle}
    \begin{columns}
    \begin{column}{0.5\textwidth}
        \begin{itemize}
            \item<1-> Descripción (\textit{qtext})
            \item<2-> Pistas (\textit{qtext})
            \item<3-> Solución (\textit{script/qscript})
            \item<4-> Confirmación (\textit{qtext})
        \end{itemize}
    \end{column}
    \begin{column}{0.5\textwidth}
        \only<1>{\includegraphics[width=0.7\textwidth]{imgs/reto1_1.png}}
        \only<2>{\includegraphics[width=0.7\textwidth]{imgs/reto1_2.png}}
        \only<3>{\includegraphics[width=0.7\textwidth]{imgs/reto1_3.png}}
        \only<4>{\includegraphics[width=0.7\textwidth]{imgs/reto1_4.png}}
    \end{column}
    \end{columns}

    \note[itemize]{
        \item Primeros 4 bytes es el tamaño del archivo
        \item Se compone de múltiples comandos + argumentos
        \item Después de \texttt{0x0000} va el ID del comando
        \item El formato de los argumentos es tipo + valor
        \item El tipo \texttt{0x0001} es un entero de 32 bits
        \item El tipo \texttt{0x0003} es una cadena de caracteres de longitud variable
        \item Los scripts terminan con \texttt{0x000C}
    }
\end{frame}
